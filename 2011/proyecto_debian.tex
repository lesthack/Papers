\documentclass{beamer}

\mode<presentation> {
  \usetheme{Darmstadt}
  \setbeamercovered{transparent}
}

\usepackage[spanish]{babel}
\usepackage[utf8]{inputenc}
\usepackage{pdfpages}
\usepackage{alltt}
\usepackage{verbatim}
\usepackage{hyperref}

%% ulem nos da mayor control de subrayados; \normalem evita que todos
%% los \em sean subrayados
\usepackage{ulem}
\normalem

\newcommand{\confurl}{\url{http://lesthack.com.mx}}
\newcommand{\authdata}{Jorge Luis Hernández — {\tt lesthack@gmail.com}}

\title{El Proyecto Debian GNU/Linux}

\author{\authdata \\ \confurl}

\institute[S3; Debian]
{Software Developer And Debian Friend}

\pgfdeclareimage[height=2cm]{debian-logo}{debian-swirl}
\logo{\pgfuseimage{debian-logo}}

\date[México, 17/09/2011]{Software Freedom Day\\
  Guanajuato, Septiembre 2011}

\AtBeginSection[]
{
  \begin{frame}<beamer>
    \frametitle{Índice Temático}
    \tableofcontents[currentsection]
  \end{frame}
}

% If you wish to uncover everything in a step-wise fashion, uncomment
% the following command: 
%\beamerdefaultoverlayspecification{<+->}

\begin{document}

\begin{frame}
  \titlepage
\end{frame}

%%%%%%%%%%%%%%%%%%%%%%%%%%%%%%%%%%%%%%%%%%%%%%%%%%
\section[¿Qué es Debian?]{¿Qué es Debian?}

%%%%%%%%%%%%%%%%%%%%%%%%%%%%%%%%%%%%%%%%%%%%%%%%%%
\begin{frame}
  \frametitle {¿Qué es Debian?}
  \begin{itemize}
  \item Una distribución de Software Libre basada en Linux
  \item Un proyecto social
    \begin{itemize}
    \item Asociación de personas que han hecho causa común para crear un sistema operativo libre.
    \item Basado en el kernel Linux.
      \begin{itemize}
      \item Aunque existe también una implementación del kernel Hurd.
      \end{itemize}    
    \item Integra muchas herramientas del proyecto GNU.
    \end{itemize}
    \item Extremistas en Control de Calidad
      \begin{itemize}
      \item Excelente Metodología de Trabajo Distribuido.
      \item Alrededor de 880 (aprox) Debian Developers.
      \item Y 120 (aprox) Debian Maintainers.
      \end{itemize}
  \end{itemize}
\end{frame}

%%%%%%%%%%%%%%%%%%%%%%%%%%%%%%%%%%%%%%%%%%%%%%%%%%
\begin{frame}
  \frametitle {El {\em Sistema Operativo Universal}}
    \begin{itemize}
    \item Correr en la mayoría del hardware posible.
      \begin{itemize}
      \item Un conjunto de Software Libre que busca ser útil para toda
      necesidad, sobre cualquier pedazo de hardware posible.
      \end{itemize}
    \item Distribución Base para Distribuciones Derivadas o Mezclas Puras.
  \end{itemize}
\end{frame}

%%%%%%%%%%%%%%%%%%%%%%%%%%%%%%%%%%%%%%%%%%%%%%%%%%
\begin{frame}
  \frametitle {Debian en el Mundo}
  \begin{center}
    Debian es la interacción de miles de personas 
    \\distribuídas por todo el mundo.
    \includegraphics[height=14em]{developers_map.jpg}\\
    {\scriptsize (incluye {\em ruido}, no es un mapa fidedigno)}
  \end{center}
\end{frame}

%%%%%%%%%%%%%%%%%%%%%%%%%%%%%%%%%%%%%%%%%%%%%%%%%%
\begin{frame}
  \frametitle {Un poco de Historia}
  \begin{itemize}
    \item Nace el 16 de Agosto de 1993 (ya es mayor de edad !!)
    \item Ian Murdock fundador y líder del proyecto.
    \item El proyecto nace con el concepto de “Distribución”.
    \item El primer año fue patrocinado por la FSF y el proyecto GNU.
    \item Fue pensada para:
      \begin{itemize}
        \item Ser desarrollada cuidadosamente.
        \item Y ser mantenida y soportada de forma similar.
      \end{itemize}
  \end{itemize}
\end{frame}

%%%%%%%%%%%%%%%%%%%%%%%%%%%%%%%%%%%%%%%%%%%%%%%%%%
\begin{frame}
  \frametitle {Aspectos que Molan}
  \begin{itemize}
    \item Sistema “Stable as a rock”.
    \item Es una distro abierta a las contribucciones de cada desarrollador y usuarios que deseen participar con su trabajo.
    \item Es una de las pocas distros que NO es una entidad comercial.
    \item Es un gran proyecto fundamentado y constituido por: 
      \begin{itemize}
      \item Un contrato social  
      \item Documento de directrices que lo organizan.
      \end{itemize}
    \item Implementa un sistema inteligente de control de paquetes y dependencias.
  \end{itemize}
\end{frame}

%%%%%%%%%%%%%%%%%%%%%%%%%%%%%%%%%%%%%%%%%%%%%%%%%%
\begin{frame}
  \frametitle {Aspectos que No Molan}
  \begin{itemize}
    \item Ciclos de Desarrollos Lentos
    \item Poco pulido y no tan atractivo
  \end{itemize}
\end{frame}

%%%%%%%%%%%%%%%%%%%%%%%%%%%%%%%%%%%%%%%%%%%%%%%%%%
\begin{frame}
  \frametitle {Los Nombres Clave de Debian}
  \begin{itemize}
    \item Debian 1.1 (Buzz) :: 17 de junio de 1996
    \item Debian 1.2 (Rex) :: 12 de diciembre de 1996
    \item Debian 1.3 (Bo) :: 2 de Junio de 1997
    \item Debian 2.0 (Hamm) :: 24 de Julio de 1998
    \item Debian 2.1 (Slinky) :: 9 de Marzo de 1999
    \item Debian 2.2 (Potato) :: 15 de Agosto de 2000
    \item Debian 3.0 (Woody) :: 19 de Julio de 2002
    \item Debian 3.1 (Sarge) :: 6 de Junio de 2005
  \end{itemize}
\end{frame}

%%%%%%%%%%%%%%%%%%%%%%%%%%%%%%%%%%%%%%%%%%%%%%%%%%
\begin{frame}
  \frametitle {Los Nombres Clave de Debian}
  \begin{itemize}
    \item Debian 4.0 (Etch) ::  8 de Abril de 2007
    \item Debian 5.0 (Lenny) :: 14 de Febrero de 2009
    \pause
    \item Debian 6.0 (Squeeze) [Stable] :: 6 de Febrero de 2011
  \end{itemize}
\end{frame}

%%%%%%%%%%%%%%%%%%%%%%%%%%%%%%%%%%%%%%%%%%%%%%%%%%
\begin{frame}
  \frametitle {Las Ramas}
  \begin{itemize}
    \item Stable (Debian Squeeze)
    \item Testing (Debian Wheeze) ?
    \item Unstable (Debian Sid)
    \item Experimental
    \item Backports
    \pause
    \item main, contrib, non-free
  \end{itemize}
\end{frame}

%%%%%%%%%%%%%%%%%%%%%%%%%%%%%%%%%%%%%%%%%%%%%%%%%%
\begin{frame}
  \frametitle {Documentos fundacionales}
  \begin{center}
    Dos de los documentos fundacionales de Debian determinan su
    identidad; determinan quién lo conforma, y a qué se compromete:
  \end{center}
  \begin{enumerate}
  \item El Contrato Social (SC)
  \item Las Directrices de Software Libre de Debian (DFSG)
  \end{enumerate}
\end{frame}
%%%%%%%%%%%%%%%%%%%%%%%%%%%%%%%%%%%%%%%%%%%%%%%%%%
\begin{frame}
  \frametitle {El Contrato Social (SC)}
  \begin{center}
    ¿A qué se compromete Debian ante sus usuarios?
  \end{center}
  \begin{itemize}
    \item Debian permanecerá siendo 100\% Software Libre 
    \item Contribuiremos de vuelta a la comunidad del Software Libre
    \item No esconderemos problemas
    \item Nuestras prioridades son nuestros usuarios y el Software Libre 
    \item Programas que no cumplen nuestros lineamientos de qué es
      Software Libre pero son libremente redistribuíbles {\em pueden}
      formar parte de nuestro archivo y usar {\em parte} de nuestra
      infraestructura, pero en una sección que indica claramente que
      no son libres, y {\em no serán considerados parte de Debian}
  \end{itemize}
\end{frame}

%%%%%%%%%%%%%%%%%%%%%%%%%%%%%%%%%%%%%%%%%%%%%%%%%%
\begin{frame}
  \frametitle {Directrices de Software Libre de Debian (DFSG)}
  \begin{center}
    ¿Y qué reconoce Debian como Software Libre?
  \end{center}
  \begin{itemize}
    \item Libre Distribución
    \item Código Fuente
    \item Trabajos Derivados
    \item Integridad del código del autor
    \item No discriminación contra personas o grupos
    \item No discriminación en función de la finalidad perseguida
    \item Distribución de la Licencia
    \item La licencia no ha de ser especifica para Debian
    \item La licencia no debe contaminar a otros programas
    \item Ejemplos de licencias: GPL, BSD, Artistica
  \end{itemize}
\end{frame}

%%%%%%%%%%%%%%%%%%%%%%%%%%%%%%%%%%%%%%%%%%%%%%%%%%
\begin{frame}
  \frametitle {¿Qué lleva a una persona a comprometerse con Debian?}
  \begin{center}
    Ser reconocido como miembro del proyecto (DD) es un proceso largo
    y complejo. ¿Qué ha llevado a tanta gente a seguirlo?
  \end{center}
  \begin{itemize}
  \item Identificación con los fines del proyecto
  \item Necesidad de contribuir con {\em algún} proyecto libre
  \item Incidir en o mejorar al rendimiento de Debian en determinado
    campo de aplicación
  \item Impulsar un proyecto personal incluyéndolo en una distribución
    relevante
  \end{itemize}
\end{frame}

%%%%%%%%%%%%%%%%%%%%%%%%%%%%%%%%%%%%%%%%%%%%%%%%%%
\section[Colaboración]{Colaboración y Estructura Social}

%%%%%%%%%%%%%%%%%%%%%%%%%%%%%%%%%%%%%%%%%%%%%%%%%%
\begin{frame}
  \frametitle {Niveles de Contribución}
  \begin{itemize}
  \item Desarrollador (DD) (+proceso NM)
  \item Mantenedor (DM)
  \item Mantenedor (Con sponsors)
  \item Traductor, Documentador
  \item Artista
  \item Abogados, Asesores
  \item ¿ Como permitir la participación directa, como miembros con todos los derechos, a gente no técnica ?
  \end{itemize}
\end{frame}

%%%%%%%%%%%%%%%%%%%%%%%%%%%%%%%%%%%%%%%%%%%%%%%%%%
\begin{frame}
  \begin{center}
  \frametitle {Niveles de Contribución}
  \includegraphics[width=28em]{debian-organigram.png}
  \end{center}
\end{frame}

%%%%%%%%%%%%%%%%%%%%%%%%%%%%%%%%%%%%%%%%%%%%%%%%%%
\begin{frame}
  \frametitle {¿Y qué lleva a una persona a preferir {\em utilizar}
    Debian?}
  \begin{center}
    Por otro lado, por qué {\em prefieres utilizar} Debian?

    Podemos resumir esto en tres grandes rubros
  \end{center}
  \begin{itemize}
  \item Factores técnicos
  \item Factores ideológicos
  \item Factores pragmáticos
  \end{itemize}
\end{frame}

%%%%%%%%%%%%%%%%%%%%%%%%%%%%%%%%%%%%%%%%%%%%%%%%%%
\begin{frame}
  \frametitle {Factores técnicos}
  \begin{itemize}
  \item Ideas inovadoras introducidas por Debian (p.ej. {\tt apt}),
    configuración orientada a directorios, {\em instale y use}, etc.
  \item Componentes de infraestructura: lintian, buildd, debhelper
  \item Multiples Alternativas al mismo problema
  \end{itemize}
\end{frame}

%%%%%%%%%%%%%%%%%%%%%%%%%%%%%%%%%%%%%%%%%%%%%%%%%%
\begin{frame}
  \frametitle {Factores ideológicos}
  \begin{itemize}
  \item Una de las ya muy pocas distribuciones 100\% impulsadas por
    una comunidad
    \begin{itemize}
      \item En contraposición del impulso de una empresa
      \item Factores de libertad del proyecto respecto a una agenda
      \item Viabilidad a largo plazo
      \item Baja relevancia de la {\em cuota de mercado}
    \end{itemize}
  \item Posición coherente y sin compromisos respecto a la libertad
    del software
    \begin{itemize}
      \item Limpieza del kernel de {\em blobs binarios} — Una tarea
        ardua, ingrata, sin ninguna {\em ventaja comercial} realizada
        a lo largo de años
      \item Posición consistente respecto a patentes de software,
        codecs e intérpretes propietarios
      \item Renuencia a incluir software restringido, sin importar que
        sea de alta demanda — Logrando en varios casos incidir en su
        relicenciamiento como Software Libre
    \end{itemize}
  \end{itemize}
\end{frame}

%%%%%%%%%%%%%%%%%%%%%%%%%%%%%%%%%%%%%%%%%%%%%%%%%%
\begin{frame}
  \frametitle {Factores pragmáticos}
  \begin{itemize}
  \item Es la única distribución que funciona del mismo modo en todos
    mis dispositivos (multi-arquitectura)
  \item Rendimiento aceptable en sistemas de bajos recursos
  \item Garantía de poder derivar de/reutilizar al 100\% de la
    distribución
  \item Mera disponibilidad de decenas de miles de paquetes
    independientes (29,000)
  \end{itemize}
\end{frame}

%%%%%%%%%%%%%%%%%%%%%%%%%%%%%%%%%%%%%%%%%%%%%%%%%%
\begin{frame}
  \frametitle {Proyectos derivados}
  \begin{itemize} 
  \item Numerosos proyectos a lo largo de los años se han basado en
    Debian, para evitar {\em reinventar la rueda} y para aprovechar
    control de calidad, seguridad
  \item Distribuciones de todo tipo — Nacionales (LinEx, Limux,
    GuadaLinex…), comerciales (Progeny, Corel, Xandros, Ubuntu…),
    comunitarias con un enfoque específico (Skolelinux, Quantian,
    Trisquel, …)
  \item Debian busca y fomenta este tipo de especialización, creando
    los {\em marcos} para su organización como {\em Custom Debian
      Distribution} o {\em Pure Debian Blend}
  \end{itemize}
\end{frame}

%%%%%%%%%%%%%%%%%%%%%%%%%%%%%%%%%%%%%%%%%%%%%%%%%%
\begin{frame}
  \frametitle {Integrando los esfuerzos de terceros}
  \begin{itemize} 
  \item Buscamos que a estas distribuciones derivadas les sea fácil
    contribuir de vuelta, reducir las divergencias — Y en líneas
    generales, lo logramos
  \item Debian ha logrado evitar la explosión sin regreso que marcó a
    las distribuciones derivadas de RedHat en los 1990s, buscando
    ofrecer las herramientas para una ágil reintegración
  \item Incluso en el caso de las distribuciones con divergencias más
    pronunciadas y frecuentes (como Ubuntu), con el paso de los años
    hemos ido encontrando canales adecuados para colaborar
  \end{itemize}
\end{frame}

%%%%%%%%%%%%%%%%%%%%%%%%%%%%%%%%%%%%%%%%%%%%%%%%%%
\section[QA]{Control de Calidad}

%%%%%%%%%%%%%%%%%%%%%%%%%%%%%%%%%%%%%%%%%%%%%%%%%%
\begin{frame}
  \frametitle {Debian y el control de calidad}
  \begin{itemize} 
    \item Debian tiene la fase de pruebas más prolongada y estricta de
      todas las distribuciones. 
    \item Miles de usuarios prueban el software y reportan fallos
      antes de liberar una nueva versión. 
    \item Hoy por hoy Debian es, sin duda, la distribución de Linux
      que más cuida el control de calidad 
      \begin{itemize}
      \item Llegando a niveles que muchos tildan de absurdos
      \item Exigimos {\em cero bugs graves} para la liberación de una
        nueva versión
      \item Muchas importantes decisiones se toman {\em por consenso},
        y muchas veces es dificil llegar a él
      \item Nos ha llevado a tiempos de liberación de hasta tres
        años. 
      \end{itemize}
  \end{itemize}
\end{frame}

%%%%%%%%%%%%%%%%%%%%%%%%%%%%%%%%%%%%%%%%%%%%%%%%%%
\begin{frame}
  \frametitle {Ciclo de Empaquetado}
  \begin{itemize} 
  \item El Upstream libera el Software
  \item El mantenedor lo revisa, añade parches y archivos de control (empaqueta).
  \item buildd lo construye para las diferentes arquitecturas
  \item unstable - testing (estabilicación) - stable (liberación)
  \item No es magia y no se hace solo
  \pause
  \item Y no es tan sencillo
  \end{itemize}
\end{frame}

%%%%%%%%%%%%%%%%%%%%%%%%%%%%%%%%%%%%%%%%%%%%%%%%%%
\begin{frame}
  \begin{center}
  \frametitle {Ciclo de Empaquetado}
  \includegraphics[width=25em]{debian_package_cycle.png}
  \end{center}
\end{frame}

%%%%%%%%%%%%%%%%%%%%%%%%%%%%%%%%%%%%%%%%%%%%%%%%%%
\begin{frame}
  \frametitle {BTS Bug Tracking System}
  \begin{itemize}  
  \item Un sistema tan complejo debe llevar un registro de todo el proceso
  \item No remplaza el BTS del Upstream, pero si interactua con el
  \item Es un complejo sistema controlado por mails
    \begin{itemize}
    \item Mensajes entre quien reporta, el sistema y el mantenedor del paquete
    \item Etiquetas, gravedad, suscripciones    
    \end{itemize}
  \item reportbug-ng por si le temes al correo electrónico
  \end{itemize}
\end{frame}



%%%%%%%%%%%%%%%%%%%%%%%%%%%%%%%%%%%%%%%%%%%%%%%%%%
\section[Traducción]{Ciclo de Traducción}

%%%%%%%%%%%%%%%%%%%%%%%%%%%%%%%%%%%%%%%%%%%%%%%%%%
\begin{frame}
  \begin{itemize}
  \item Un robot atiende los correos enviados a la lista de traducción.
  \item El robot supervisa los correos enviados y gestiona el siguiente paso.
  \item El traductor indica que se va a hacer cargo de una traducción.
  \item El traductor envia la traducción para que sea revisada.
  \item Una vez revisada, la traducción es enviada al BTS para ser incorporada.
  \item El mantenedor incorpora la traducción, la errata y la cierra.
  \end{itemize}
\end{frame}

%%%%%%%%%%%%%%%%%%%%%%%%%%%%%%%%%%%%%%%%%%%%%%%%%%
\begin{frame}
  \frametitle {ITT - Intent to Translate (Intento de Traducción)}
  \begin{itemize}
  \item Se envía para indicar que se va a trabajar en una traducción, evitando la duplicación de esfuerzos.  
  \item Tipos:
    \begin{itemize}
      \item po-debconf://nombrepaquete 
      \item po://nombrepaquete/rutapaquetefuente
      \item debian-installer://nombrepaquete/rutapaquetefuente 
      \item wml://direcciónpágina 
      \item man://nombrepaquete/sección/tema 
    \end{itemize}
  \item Mala Forma: 
    \begin{itemize}
    \item Subject: Me hago cargo de la traducción po-debconf de exim4 
    \end{itemize}
  \item Forma indicada:
     \begin{itemize}
    \item Subject: [ITT] po-debconf://exim4 
    \end{itemize}
  \end{itemize}
\end{frame}

%%%%%%%%%%%%%%%%%%%%%%%%%%%%%%%%%%%%%%%%%%%%%%%%%%
\begin{frame}
  \frametitle {RFR - Request For Review (Solicitud de revisión)}
  \begin{itemize}
  \item La traducción se ha terminado y, adjunto al correo, se envíe para que otras personas puedan revisarla y detectar errores.
  \item Le sigue en algunos casos un RFR2 si se realizan cambios sustanciales.
  \end{itemize}
\end{frame}

%%%%%%%%%%%%%%%%%%%%%%%%%%%%%%%%%%%%%%%%%%%%%%%%%%
\begin{frame}
  \frametitle {LCFC - Last Chance For Comments (última oportunidad)}
  \begin{itemize}
  \item Indica que la traducción se ha terminado, se han incorporado los cambios del proceso de revisión y se enviará en breve al lugar apropiado.
  \item Se pueden enviar cuando no hay ITR
  \end{itemize}
\end{frame}

%%%%%%%%%%%%%%%%%%%%%%%%%%%%%%%%%%%%%%%%%%%%%%%%%%
\begin{frame}
  \frametitle {BTS - Bug Tracking System}
  \begin{itemize}
  \item Indica que se ha registrado una errata con respecto a la traducción en el BTS.
  \item Cada día el robot comprobará si el informe de error sigue o no abierto.
  \end{itemize}
\end{frame}

%%%%%%%%%%%%%%%%%%%%%%%%%%%%%%%%%%%%%%%%%%%%%%%%%%
\begin{frame}
  \frametitle {DONE - Hecho}
  \begin{itemize}
  \item Indica que la traducción ya se ha hecho.
  \item O bien se ha arreglado la errata enviada al BTS.
  \item O bien la traducción ya no aplica (el paquete se ha borrado, se ha abandonado el trabajo, etc.).
  \end{itemize}
\end{frame}


%%%%%%%%%%%%%%%%%%%%%%%%%%%%%%%%%%%%%%%%%%%%%%%%%%
\begin{frame}
  \frametitle{¿Dudas?}
  \vskip 2em
  {\Large ¡Gracias!}
  \vskip 1em
  \authdata \\ \confurl
\end{frame}

\end{document}
